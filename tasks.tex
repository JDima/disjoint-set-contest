\documentclass[a4]{article}
\pagestyle{myheadings}

%%%%%%%%%%%%%%%%%%%
% Packages/Macros %
%%%%%%%%%%%%%%%%%%%
\usepackage{mathrsfs}


\usepackage{fancyhdr}
\pagestyle{fancy}
\lhead{}
\chead{}
\rhead{}
\lfoot{}
\cfoot{} 
\rfoot{\normalsize\thepage}
\renewcommand{\headrulewidth}{0pt}
\renewcommand{\footrulewidth}{0pt}
\newcommand{\RomanNumeralCaps}[1]
    {\MakeUppercase{\romannumeral #1}}

\usepackage{amssymb,latexsym}  % Standard packages
\usepackage[utf8]{inputenc}
\usepackage[russian]{babel}
\usepackage{MnSymbol}
\usepackage{mathrsfs}
\usepackage{amsmath,amsthm}
\usepackage{indentfirst}
\usepackage{graphicx}%,vmargin}
\usepackage{graphicx}
\graphicspath{{pictures/}} 
\usepackage{verbatim}
\usepackage{color}
\usepackage{color,colortbl}
\usepackage[nottoc,numbib]{tocbibind}
\usepackage{float}
\usepackage{multirow}
\usepackage{hhline}

\usepackage{listings}
\definecolor{codegreen}{rgb}{0,0.6,0}
\definecolor{codegray}{rgb}{1,1,1}
\definecolor{codepurple}{rgb}{0.58,0,0.82}
\definecolor{backcolour}{rgb}{0.95,0.95,0.92}
 
\lstdefinestyle{mystyle}{
    backgroundcolor=\color{backcolour},   
    commentstyle=\color{codegreen},
    keywordstyle=\color{magenta},
    numberstyle=\tiny\color{codegray},
    stringstyle=\color{codepurple},
    basicstyle=\footnotesize,
    breakatwhitespace=false,         
    breaklines=true,                 
    captionpos=b,                    
    keepspaces=true,                 
    numbers=left,                    
    numbersep=5pt,                  
    showspaces=false,                
    showstringspaces=false,
    showtabs=false,                  
    tabsize=2
}
 
\lstset{style=mystyle}

\usepackage{url}
\urldef\myurl\url{foo%.com}
\def\UrlBreaks{\do\/\do-}
\usepackage{breakurl}
\Urlmuskip=0mu plus 1mu



\DeclareGraphicsExtensions{.pdf,.png,.jpg}% -- настройка картинок

\usepackage{epigraph} %%% to make inspirational quotes.
\usepackage[all]{xy} %for XyPic'a
\usepackage{color} 
\usepackage{amscd} %для коммутативных диграмм
%\usepackage[colorlinks,urlcolor=red]{hyperref}

%\renewcommand{\baselinestretch}{1.5}
%\sloppy
%\usepackage{listings}
%\lstset{numbers=left}
%\setmarginsrb{2cm}{1.5cm}{1cm}{1.5cm}{0pt}{0mm}{0pt}{13mm}


\newtheorem{Lemma}{Лемма}[section]
\newtheorem{Proposition}{Предложение}[section]
\newtheorem{Theorem}{Теорема}[section]
\newtheorem{Corollary}{Следствие}[section]
\newtheorem{Remark}{Замечание}[section]
\newtheorem{Definition}{Определение}[section]
\newtheorem{Designations}{Обозначение}[section]




%%%%%%%%%%%%%%%%%%%%%%% 
%Подготовка оглавления% 
%%%%%%%%%%%%%%%%%%%%%%% 
\usepackage[titles]{tocloft}
\renewcommand{\cftdotsep}{2} %частота точек
\renewcommand\cftsecleader{\cftdotfill{\cftdotsep}}
\renewcommand{\cfttoctitlefont}{\hspace{0.38\textwidth} \LARGE\bfseries} 
\renewcommand{\cftsecaftersnum}{.}
\renewcommand{\cftsubsecaftersnum}{.}
\renewcommand{\cftbeforetoctitleskip}{-1em} 
\renewcommand{\cftaftertoctitle}{\mbox{}\hfill \\ \mbox{}\hfill{\footnotesize Стр.}\vspace{-0.5em}} 
%\renewcommand{\cftchapfont}{\normalsize\bfseries \MakeUppercase{\chaptername} } 
%\renewcommand{\cftsecfont}{\hspace{1pt}} 
\renewcommand{\cftsubsecfont}{\hspace{1pt}} 
%\renewcommand{\cftbeforechapskip}{1em} 
\renewcommand{\cftparskip}{3mm} %определяет величину отступа в оглавлении
\setcounter{tocdepth}{5} 
\renewcommand{\listoffigures}{\begingroup %добавляем номер в список иллюстраций
\tocsection
\tocfile{\listfigurename}{lof}
\endgroup}
\renewcommand{\listoftables}{\begingroup %добавляем номер в список иллюстраций
\tocsection
\tocfile{\listtablename}{lot}
\endgroup}


%\renewcommand{\thelikesection}{(\roman{likesection})}
%%%%%%%%%%%
% Margins %
%%%%%%%%%%%
\addtolength{\textwidth}{0.7in}
\textheight=630pt
\addtolength{\evensidemargin}{-0.4in}
\addtolength{\oddsidemargin}{-0.4in}
\addtolength{\topmargin}{-0.4in}

%%%%%%%%%%%%%%%%%%%%%%%%%%%%%%%%%%%
%%%%%%Переопределение chapter%%%%%% 
%%%%%%%%%%%%%%%%%%%%%%%%%%%%%%%%%%%
\newcommand{\empline}{\mbox{}\newline} 
\newcommand{\likechapterheading}[1]{ 
\begin{center} 
\textbf{\MakeUppercase{#1}} 
\end{center} 
\empline} 

%%%%%%%Запиливание переопределённого chapter в оглавление%%%%%% 
\makeatletter 
\renewcommand{\@dotsep}{2} 
\newcommand{\l@likechapter}[2]{{\bfseries\@dottedtocline{0}{0pt}{0pt}{#1}{#2}}} 
\makeatother 
\newcommand{\likechapter}[1]{ 
\likechapterheading{#1} 
\addcontentsline{toc}{likechapter}{\MakeUppercase{#1}}} 




\usepackage{xcolor}
\usepackage{hyperref}
\definecolor{linkcolor}{HTML}{000000} % цвет ссылок
\definecolor{urlcolor}{HTML}{AA1622} % цвет гиперссылок
 
\hypersetup{pdfstartview=FitH,  linkcolor=linkcolor,urlcolor=urlcolor, colorlinks=true}

%%%%%%%%%%%%
% Document %
%%%%%%%%%%%%

%%%%%%%%%%%%%%%%%%%%%%%%%%%%%
%%%%%%главы -- section*%%%%%%
%%%%section -- subsection%%%%
%subsection -- subsubsection%
%%%%%%%%%%%%%%%%%%%%%%%%%%%%%
\def \newstr {\medskip \par \noindent} 
\begin{document}



\section*{Задача 1}
\label{sec:orgb62fe60}
\subsection*{Постановка}
\label{sec:org37954e9}
Существует алфавит размера \(m\). Сколько можно построить строк длины \(n\), чтобы любая подстрока длины \(k\) являлась палиндромом. Подстрока - это палиндромом, если она
одинаково читается как слева направо, так и справа налево.

\subsection*{Входные данные}
\label{sec:orgc51833b}
Строка, содержащая три целых числа: \(n\),\(m\),\(k\).

\subsection*{Выходные данные}
\label{sec:org91cd1c2}
Число равное количеству строк.

\subsection*{Пример 1}
\label{sec:org1b720b0}

\begin{table}[H]
\begin{center}
\begin{tabular}{|m{4cm}|m{4cm}|}
\hline
Входные данные & Выходные данные \\ \hline
1 1 1
&
1
\\ \hline
\end{tabular}
\end{center}
\end{table}

\subsection*{Пример 2}
\label{sec:org2aeecb4}

\begin{table}[H]
\begin{center}
\begin{tabular}{|m{4cm}|m{4cm}|}
\hline
Входные данные & Выходные данные \\ \hline
5 2 4
&
2
\\ \hline
\end{tabular}
\end{center}
\end{table}

\pagebreak
\section*{Задача 2}
\label{sec:orgef181bd}
\subsection*{Постановка}
\label{sec:orgad8a20e}

В классе учатся \(n\) учеников. Учитель физкультуры поставил учеников в ряд. Силовые показвтели \(i\)-ого ученика равны \(a_i\).
\label{sec:orgb72ba50}
Группа - не пустой непрерывный отрезок этого ряда. Силой группы является минимальный силовой показатель ученика в этой группе. \\
Учителю нужно узнать максимальную силу группы размером \(x\), для всех \(x\) $\leq$ \(n\).
\subsection*{Входные данные}
\label{sec:orgc51833b}
В первой строке ввода записано целое число \(n\), количество учеников.
\label{sec:orgc51833b}
Во второй строке записано \(n\) целых чисел \(a_1\), \(a_2\), ..., \(a_n\) — силовых показателей.
\subsection*{Выходные данные}
\label{sec:orgf9da829}

Выведите \(n\) целых чисел - максимальных сил групп для заданных значений \(x\).

\subsection*{Пример}
\label{sec:orgd7d348d}

\begin{table}[H]
\begin{center}
\begin{tabular}{|m{4cm}|m{4cm}|}
\hline
Входные данные & Выходные данные \\ \hline
10

1 2 3 4 5 4 3 2 1 6
&
6 4 4 3 3 2 2 1 1 1
\\ \hline
\end{tabular}
\end{center}
\end{table}

\pagebreak
\section*{Задача 3}
\label{sec:org570b899}
\subsection*{Постановка}
\label{sec:orga2b5149}
Дана последовательность длиной \(n\). Найдите подпоследовательность длины \(k\) (\(k\)$\leq$\(n\)) с минимальной ценой.
\label{sec:orged795e8}
Цена подпоследовательности определяется как минимум между:
\begin{itemize}
    \item Максимумом по числам, стоящим на нечетных позициях.
    \item Максимумом по числам, стоящим на четных позициях.
\end{itemize}
\subsection*{Входные данные}
\label{sec:orgeb4908d}
В первой строке записаны два ценлых числа \(n\) - длина последовательности и \(k\) - длина подпоследовательности.
\label{sec:orged795e8}
Во второй строке записано \(n\) целых чисел \(a_1\), \(a_2\), ..., \(a_n\) - элементы последовательности.

\subsection*{Выходные данные}
\label{sec:orged795e8}
Выведите минимальную цену подпоследовательности размера \(k\).
\subsection*{Пример 1}
\label{sec:org6a26c04}

\begin{table}[H]
\begin{center}
\begin{tabular}{|m{4cm}|m{4cm}|}
\hline
Входные данные & Выходные данные \\ \hline
4 2

1 2 3 4
&
1 
\\ \hline
\end{tabular}
\end{center}
\end{table}

\subsection*{Пример 2}
\label{sec:orge96f7c4}

\begin{table}[H]
\begin{center}
\begin{tabular}{|m{4cm}|m{4cm}|}
\hline
Входные данные & Выходные данные \\ \hline
6 4

5 3 50 2 4 5
&
3
\\ \hline
\end{tabular}
\end{center}
\end{table}

\subsection*{Пример 3}
\label{sec:orge96f7c4}

\begin{table}[H]
\begin{center}
\begin{tabular}{|m{4cm}|m{4cm}|}
\hline
Входные данные & Выходные данные \\ \hline
4 3

1 2 3 4
&
2
\\ \hline
\end{tabular}
\end{center}
\end{table}

\pagebreak
\section*{Задача 4}
\label{sec:orgb1f46a6}
\subsection*{Постановка}
\label{sec:orge854c50}
В электрической цепи последовательно установлены \(n\) ключей. Ключ изначально имеет одно из
состояний: разомкнут (0) или замкнут (1).
\\
Дано \(k\) подмножеств \(A_1\), \(A_2\), ..., \(A_k\) множества ключей, таких что пересечение любых трех подмножеств будет пустым множеством.
\\
Можно взять одно из \(k\) подмножеств и изменить состояние всех ключей из этого подмножества на противоположное. Гарантируется, что для данных подмножеств можно совершить несколько операций так, чтобы цепь замкнулась (все ключи будут в состоянии 1).
\\
Обозначим за \(m_i\) минимальное количество операций, которое вы должны совершить,
чтобы первые \(i\) ключей оказались замкнутыми. Обратите внимание, что при этом состояние
других ключей(с номерами между \(i+1\) и \(n\)) может быть любым.\\
Необходимо посчитать минимальное количество операций которое нужно совершить,
чтобы первые \(i\) ключей оказались замкнутыми для всех \(i\) (1 $\leq$ \(i\) $\leq$ \(n\))


\subsection*{Входные данные}
\label{sec:orge854c50}
В первой строке \(n\) и \(k\).
\\
Во второй строке записаны начальные состояния всех ключей.
\\
Далее следуют описания \(k\) подмножеств:
\\
\begin{itemize}
    \item В первой строке находится целое число \(c\) (1 $\leq$ \(c\) $\leq$ \(n\)) — количество элементов в
подмножестве.
    \item Во второй строке находится\(c\) целых чисел \(x_1\), . . . , \(x_c\) (1 $\leq$ \(x_i\) $\leq$ \(n\)) - элементы подмножества.
\end{itemize}

\subsection*{Выходные данные}
\label{sec:org1ab7414}
Необходимо через пробел вывести минимальные количества операций необходимых
для того чтобы включить лампы от 1 до \(i\) для всех \(i\).

\subsection*{Пример}
\label{sec:org25482f8}

\begin{table}[H]
\begin{center}
\begin{tabular}{|m{4cm}|m{4cm}|}
\hline
Входные данные & Выходные данные \\ \hline
7 3

0011100

3

1 4 6

3

3 4 7

2

2 3
&
1 2 3 3 3 3 3
\\ \hline
\end{tabular}
\end{center}
\end{table}

\begin{table}[H]
\begin{center}
\begin{tabular}{|m{4cm}|m{4cm}|}
\hline
Входные данные & Выходные данные \\ \hline
5 3

00011

3

1 2 3

1

4

3

3 4 5
&
1 1 1 1 1
\\ \hline
\end{tabular}
\end{center}
\end{table}
\begin{table}[H]
\begin{center}
\begin{tabular}{|m{4cm}|m{4cm}|}
\hline
Входные данные & Выходные данные \\ \hline
19 5

1001001001100000110

2

2 3

2

5 6

2

8 9

5

12 13 14 15 16

1

19
&
0 1 1 1 2 2 2 3 3 3 3 4 4 4 4 4 4 4 5
\\ \hline
\end{tabular}
\end{center}
\end{table}
\pagebreak
\end{document}
